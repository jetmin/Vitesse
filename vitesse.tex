\documentclass[titlepage,a4paper,12pt]{article}

\usepackage{amsmath,amsfonts,amssymb,amsthm,mathrsfs,enumitem}
\usepackage[french]{babel}
\usepackage{tikz}
\usepackage[utf8]{inputenc}
\usepackage{centernot}
\usepackage[parfill]{parskip}
\newcounter{def}
\newcounter{thm}
\newcounter{prop}
\newcounter{cor}
\frenchbsetup{StandardLists = true}


\newtheorem{defi}[def]{Définition}
\newtheorem{thm}[thm]{Théorème}
\newtheorem{propo}[prop]{Proposition}
\newtheorem{cor}[cor]{Corollaire}

\newcommand{\nlongleftrightarrow}{\longleftrightarrow \kern-15pt\not \kern15pt}


\setlength{\parskip}{0pt}
\begin{document}
\section{Le modèle}\emph{\color{blue}(à refaire en temps continu)}
Nous notons $(X_t)_{t\geqslant 0}$ le processus de percolation dynamique sur le réseau $(\mathbb{Z}^d,\mathbb{E}^d)$ et nous considérons la restriction de $X$ dans une boîte finie $\Lambda$. Nous définissons la suite $(\tau_i)_{i\geqslant 0}$ des instants de changement dans une boîte finie $\Lambda(\ell,h)$, en posant $\tau_0 = 0$ et pour tout $i\geqslant 0$,
$$\tau_{i+1} \,=\, \inf \,\big\{ \, t>\tau_i: {X_t}\mid_{ \Lambda(\ell,h)} \neq {X_{\tau_i}}\mid_{ \Lambda(\ell,h)} \big \}.
$$
Pour tout $i\geqslant 1$, il existe une unique arête $e_i$ incluse dans $\Lambda(\ell,h)$ telle que $X_{\tau_i}(e_i) \neq X_{\tau_{i-1}}(e_i)$. Nous appelons la suite $(e_i)_{i\geqslant 1}$ la suite des arêtes modifiées.
Nous notons $\{T\longleftrightarrow B\}$ l'événement il existe un chemin ouvert entre le haut et le bas de la boîte $\Lambda$ et nous allons coupler $(Y_t)_{t\geqslant 0}$ à valeurs dans $\{0,1\}^{\mathbb{E}^d}$. D'abord nous posons $Y_0=X_0$ et $X_0$ une configuration qui vérifie $\{T\nlongleftrightarrow B\}$. Ensuite, soit $i\geqslant 0$, pour tout $s\in [\tau_i, \tau_{i+1}[$, nous posons $Y_s = Y_{\tau_i}$, et $Y_{\tau_{i+1}}(e) =Y_{\tau_i}(e)$ si $e\neq e_{i+1}$, et nous déterminons $Y_{\tau_{i+1}}(e_{i+1})$ en fonction de $X_{\tau_{i+1}}$ via la formule suivante:
$$Y_{\tau_{i+1}}(e_{i+1})=\left\lbrace \begin{array}{cl}
0 &\text{si }X_{\tau_{i+1}}(e_{i+1}) = 0\\
1 & \text{si }X_{\tau_{i+1}}(e_{i+1}) = 1, T \nlongleftrightarrow B \text{ dans }(Y_{\tau_i})^{e_{i+1}}\\
0 & \text{si }X_{\tau_{i+1}}(e_{i+1}) = 1, T \longleftrightarrow B \text{ dans }(Y_{\tau_i})^{e_{i+1}}
\end{array}\right..$$
Le processus $Y$ est le processus de percolation dynamique conditionné à rester dans l'ensemble $\{T\nlongleftrightarrow B\}$. Nous pouvons définir l'interface à l'aide de ce couplage.
\begin{defi}
Soit $(X_t,Y_t)_{t\geqslant 0}$ un couplage défini précédemment, nous définissons l'interface dans $\Lambda({\ell,h})$ au temps $t$, que nous notons $\mathcal{I}_t({\ell,h})$, comme l'ensemble aléatoire des arêtes qui sont ouvertes dans $X_t$ et fermées dans $Y_t$: $$ \mathcal{I}_t({\ell,h}) \,=\, \big\{ \,e\in \mathbb{E}^2: X_t(e) = 1, Y_t(e) = 0 \, \big\}.
$$
\end{defi}
Nous notons $\mathcal{P}_t$ l'ensemble des arêtes pivot pour l'événement $\{T\longleftrightarrow B\}$ dans $Y$ à l'instant $t$.
\section{La construction d'un chemin espace-temps}
Nous allons énoncer une première proposition qui permet de construire un chemin espace temps dont nous avons besoin pour l'estimée.

\begin{propo} \label{stc}Soit $0<s<t$ deux instants et $e\in \mathcal{P}_t$ et $f \in \mathcal{P}_s$. Il existe un chemin espace-temps monotone en temps et fermé dans le processus $Y$ qui relie $(e,t)$ à $(f,s)$. De plus, si une arête $e$ est visitée à deux instants différents $u$ et $v$, il existe un instant $w\in ]u,v[$ où $e$ est ouverte à $w$.
\end{propo}

\begin{proof}
Nous remarquons d'abord qu'à l'instant $t$ donné, il existe un chemin fermé qui relie tous les arêtes pivot de cet instant. Nous considérons $\theta(t)$ le dernier instant où il existe une arête de $\mathcal{P}_t$ qui devient pivot, i.e.,
$$ \theta(t) = \sup \big\{ \,r\leqslant t, \exists e_1 \in \mathcal{P}_t, e_1 \text{ devient pivot à l'instat } r,e_1\text{ pivot sur }[r,t]\, \big\}.
$$
Nous montrons que $\theta(t)< t$. En effet, si à l'instant $t$, il y a une arête qui se ferme, alors il n'y a pas d'arête pivot créé à l'instant $t$, donc les arêtes de $\mathcal{P}_t$ sont aussi pivot à l'instant $t-1$. Dans le cas contraire, si une arête s'ouvre à l'instant $t$. Nous considérons une arête $\epsilon$ de $\mathcal{P}_{t-1}$, il existe un chemin ouvert passant par $\epsilon$ qui relie le haut et le bas à l'instant $t-1$. Or une ouverture d'une arête quelconque ne modifie pas ce chemin ouvert, $\epsilon$ est toujours pivot à l'instant $t$. Donc $\epsilon$ est devenu pivot avant l'instant $t$. D'où $\theta(t)< t$.
Si $\theta(t)\leqslant s$, le chemin $(e,t),(e_1,t),(e_1,s)$ est un chemin espace-temps voulu. Sinon, $\theta(t)> s$, nous considérons l'arête-temps $(e_1,\theta(t))\in \mathcal{P}_{\theta(t)}$ qui est reliée à $(e,t)$ par le chemin $(e,t),(e_1,t),(e_1,\theta(t))$. Nous devons maintenant montrer le résultat voulu avec $(e_1,\theta(t))$ à la place de $(e,t)$. Nous répétons le procédure précédent et nous construisons une suite décroissante d'instants $(t,\theta(t),\theta(\theta(t)),\dots)$ et une suite d'arête $(e,e_1,e_2,\dots)$. Or tout instant de la suite est supérieur à $s$, la suite est forcément finie. Donc il existe un $k\in\mathbb{N}$ tel que $\theta^k(t)\leqslant s$. Or pour tout $0\leqslant i < k$, l'arête $(e_i,\theta^i(t))$ est reliée à $(e_{i+1},\theta^{i+1}(t))$ par un chemin espace-temps fermé. Nous obtenons un chemin espace-temps fermé qui relie $(e,t)$ à $(e_k,\theta^k(t))$. Or $\theta^k(t)\leqslant s$ et $e_k$ est pivot entre $\theta^k(t)$ et $s$. Par construction, ce chemin est monotone en temps. 

Pour obtenir le deuxième point nous faisons un procédure de modification récursive du chemin précédent $(e_i,\theta^i(t))_{0\leqslant i \leqslant k}$. Nous commençons avec $e_1$. S'il existe $i$ une indice telle que $e_i=e_1$ et que $e_1$ est fermée entre $\theta^i(t)$ et $\theta^1(t)$, nous enlevons les arêtes entre $1$ et $i$ et nous relions directement $e_1$ et $e_i$ par un changement de temps et nous recommençons le procédure avec le nouveau chemin obtenu. Sinon, nous continuons le procédure avec $e_2$. Or la longueur du chemin diminue à chaque modification, ce procédure se termine. De plus, ce procédure ne modifie pas les l'ordre de $\theta^i$, la monotonie est conservée.
\end{proof}

\section{La comparaison entre deux processus}
Nous étudions maintenant le chemin espace-temps dans les deux processus, en particulier, nous comparons les arêtes fermées du chemin.

Soit $(e,t)$ une arête-temps fermée dans le processus $Y$. Dans le processus $X$, vu qu'il n'y a pas de conditionnement, elle peut être ouverte ou fermée à l'instant $t$. Si elle est ouverte, par la définition de l'interface, $e\in \mathcal{P}_t$. Dans le cas contraire, elle est fermée dans le processus de percolation standard, nous avons $P(X_t(e) = 0) = 1-p$. Si $e$ se ferme dans $Y$ à un instant, elle se ferme aussi dans $X$. Par contre, si $e$ est pivot dans $Y$, lors d'une arrivée du processus d'ouverture, elle s'ouvre dans $X$ mais pas dans $Y$. Maintenant, regardons le chemin espace-temps obtenu dans la proposition \ref{stc}. Or il existe des arêtes pivot dans le chemin, ce chemin n'est pas forcément fermé dans le processus $X$. Par contre, nous pouvons considérer le dernier instant de fermeture avant le passage sur une arête. Plus précisément, notons $(e_i,t_i)_{1\leqslant i \leqslant n}$ le chemin espace-temps. Nous considérons $\tilde{t_i}$ le dernier instant d'arrivée du processus de fermeture de l'arête $e_i$ avant $t_i$.\emph{\color{blue}(notation?)} Nous allons distinguer trois cas. Dans le premier cas $\tilde{t_i}\geqslant s$, l'arête $e_i$ est fermée à $t_i$ car il y a une fermeture après l'instant $s$. Dans le deuxième cas $\tilde{t_i}< s$ et $e_i$ est fermé à l'instant $s$ dans les deux processus. Dans le dernier cas, $\tilde{t_i}< s$ et à l'instant $s$, $e_i$ est fermée dans $Y$ mais ouverte dans $X$, nous remarquons que $e_i \in \mathcal{I}_s$. Nous en déduisons la proposition suivante:

\begin{propo}\label{couple} Soient $s<t$ deux instants et $\gamma$ un chemin espace-temps fermé entre $s$ et $t$ dans le processus conditionné $Y$, une arête $e$ de $\gamma$ vérifie une des trois propriétés suivantes:
\begin{itemize}[leftmargin = 0.9cm]
\item[(A)] $e\in \mathcal{I}_s$;
\item[(B)] $e$ est fermée à l'instant $s$ dans le processus de percolation dynamique $X$;
\item[(C)] Il existe un instant $r\in ]s,t]$ où il y a un instant d'arrivée du processus de fermeture.
\end{itemize}
\end{propo}


\section{Une inégalité de type BK}
Nous allons obtenir une inégalité de type BK à l'aide du couplage. Nous reprenons les notation de la proposition \ref{couple} et nous notons $\mathcal{E},\tilde{\mathcal{E}}$ et $\mathcal{D}$ les ensembles de configurations dans $\Lambda$ entre $s$ et $t$ suivants,
\begin{align*}
\mathcal{E}(\gamma)&=\{\,\gamma \text{ fermé entre }s \text{ et }t\text{ disjoint de }\mathcal{P}_s\cup \mathcal{I}_s\,\}\\
\tilde{\mathcal{E}}(\gamma)&=\{\,\text{toute arête de }\gamma\text{ vérifie } (B)\text{ ou }(C) \,\}\\
\mathcal{D}& = \{\,T\nlongleftrightarrow B \text{ entre }s \text{ et }t\,\}
\end{align*}
Nous montrons l'inégalité suivante:
\begin{propo} \label{BK'}
$P(\mathcal{E}\cap\mathcal{D})\leqslant P(\tilde{\mathcal{E}})P(\mathcal{D})$
\end{propo}
\begin{proof}
Nous utilisons le couplage et nous considérons l'événement $\mathcal{E}$ dans le processus $Y$. Or le processus $Y$ est conditionné par $\mathcal{D}$, nous obtenons:
$$P(\mathcal{E}|\mathcal{D}) = \frac{P(\mathcal{E}\cap\mathcal{D})}{P(\mathcal{D})}.
$$
Par la proposition \ref{couple}, or les arêtes du chemin espace temps sont disjointes de $\mathcal{P}_s$, elles vérifient donc $(B)$ ou $(C)$. Donc l'événement $\tilde{\mathcal{E}}$ est réalisé dans le processus $X$. Nous obtenons
$$ P(\mathcal{E}|\mathcal{D})\leqslant P(\tilde{\mathcal{E}}).
$$
Enfin, nous obtenons l'inégalité voulu en combinant les deux inégalités précédentes.
\end{proof}
Nous remarquons que cette inégalité permet de majorer l'intersection de deux événement par un produit de deux événement comme l'inégalité de BK. Mais la différence avec l'inégalité de BK est que l'un des deux événement à droite de l'inégalité est modifié selon le couplage. 
\section{La vitesse du déplacement}
Nous présentons le problème central dans cette partie. Soient $e\in \mathbb{E}^d$ une arête, $l\in \mathbb{R}$, $t\in \mathbb{N}$ un instant et $\displaystyle 0\leqslant s\leqslant K\frac{l}{\ln l}$. Soit $d$ la distance euclidienne dans $\mathbb{R}^d$ et nous supposons que $$d(e,\mathcal{P}_t\cup \mathcal{I}_t) \geqslant l.$$ Nous voulons montrer que dans ce cas, la probabilité pour que $e$ appartienne à $\mathcal{I}_t\cup \mathcal{P}_t$ décroît exponentiellement vite avec la distance $l$,i.e. 
\begin{equation}
\exists c > 0,\, P(e\in \mathcal{P}_{t+s}\cup \mathcal{I}_{t+s}|d(e,\mathcal{P}_t\cup \mathcal{I}_t) \geqslant l)\leqslant \exp(-cl)
\end{equation}
Pour montrer ce résultat, nous allons d'abord construire un chemin espace-temps selon la proposition \ref{stc} pour représenter le déplacement de l'interface. Ensuite, nous appliquons la proposition \ref{BK'} pour obtenir l'événement $\tilde{\mathcal{E}}$ dans le processus $X$. Enfin, nous estimons la probabilité de $\tilde{\mathcal{E}}$. Nous calculons d'abord $P(\tilde{\mathcal{E}})$.
\begin{propo} \label{decexp}
Soit $\gamma$ un chemin espace-temps de longueur $n$ avec $m$ changement de temps entre $s$ et $t$. Nous supposons qu'il existe une constante $\kappa$ telle que $t-s \leqslant \kappa |\lambda|$, alors il existe une constante $0\leqslant\hat{p}< 1$ et une constante $c(p)>0$ qui tend vers $+\infty$ quand $p$ tend vers $1$ telles que $$\forall p \geqslant \hat{p}, \quad P(\tilde{\mathcal{E}}(\gamma))\leqslant \exp(-c(p)(n-m)).$$
\end{propo}
\begin{proof}
Nous fixons d'abord une arête $e\in \Lambda$ et nous notons $N_e$ le nombre d'arrivées de fermeture de l'arête $e$. Nous calculons la probabilité que $e$ vérifie $(B)$ ou $(C)$.
\begin{multline*}P(e \text{ vérifie }(B) \text{ ou }(C))\leqslant P(e \text{ fermé à l'instant } s)\\+P(\exists r \in ]s,t], N_e(r)=N_e(r-1)+1)
\end{multline*}
Or la loi de la configuration dans $\Lambda$ à l'instant $s$ est la loi de la percolation Bernoulli de paramètre $p$, nous obtenons $$P(e \text{ fermé à l'instant } s) = 1-p.$$
Comme à chaque instant, nous choisissons une arête dans $\Lambda$ uniformément et nous tirons l'état de cette arête selon la loi de Bernoulli de paramètre $p$, nous obtenons la deuxième probabilité
$$ P(\exists r \in ]s,t], N_e(r)=N_e(r-1)+1) \leqslant \frac{t-s}{|\Lambda|}(1-p)\leqslant \kappa(1-p).
$$
Nous obtenons que pour une arête $e$ fixée, $$P(e \text{ vérifie }(B) \text{ ou }(C))\leqslant (1+\kappa)(1-p).$$
Nous considérons maintenant chaque arête de $\gamma$. Elle contient au moins $n-m$ déplacements en espace. Or $(B)$ et $(C)$ sont entièrement déterminés par le processus $X$ et les arêtes ne restent pas fermées entre deux visites, nous obtenons 
$$P(\tilde{\mathcal{E}}(\gamma))\leqslant (1+\kappa)^{n-m}(1-p)^{n-m} = \exp\big(\ln[(1+\kappa)(1-p)](n-m)\big).
$$ 
Nous posons $c(p) = -\ln[(1+\kappa)(1-p)]$. Or $\ln[(1+\kappa)(1-p)]$ tend vers $-\infty$ quand $p$ tend vers 1. Il existe une constante $\hat{p}<1$ telle que pour tout $p\geqslant \hat{p}$, nous avons $c(p)>0$. Nous obtenons aussi 
$ \lim_{p\rightarrow 1}c(p) = +\infty
$.
\end{proof}
Nous montrons maintenant le résultat $(1)$. 
\begin{proof}
Nous considérons d'abord le cas $e\in \mathcal{P}_{t+s}$ et est de distance au moins $l$ avec $\mathcal{P}_t$.
D'après la proposition \ref{stc}, il existe un chemin espace-temps $\rho$ qui relie $e$ à une arête de $\mathcal{P}_t$. Nous notons $\rho= (e_i,t_i)_{1\leqslant i \leqslant N}$ avec $e_1 = e, t_1 = t+s$ et $e_N\in \mathcal{P}_t, t_N = t$. Ce chemin est de projection en espace au moins $l$ car $d(e,\mathcal{P}_t)\geqslant l$. Nous notons $(e_n,t_n)$ la première arête de $\rho$ qui sort de la boule $B(e,l/2)$ de ce chemin, i.e. 
$$ n = \sup\,\big\{k>1, \forall 1<i<k, e_i\in B(e,l/2) \big\}.
$$
Nous notons $\gamma$ le bout de $\rho$ entre $(e_1,t_1)$ et $(e_n,t_n)$.
Nous écrivons donc 
$$ P(e\in \mathcal{P}_{t+s}|e\notin \mathcal{P}_t\cup \mathcal{I}_t) \leqslant \sum_{\gamma}P( e\overset{\gamma \text{ fermé}}{\longleftrightarrow} \partial B(e,l/2)).
$$
Nous fixons d'abord la longueur du chemin $n$ et le nombre de changement de temps $m$. La proposition \ref{couple} et \ref{BK'} nous permet de majorer chaque terme de la somme par $P(\tilde{\mathcal{E}}(\gamma))$ et d'après la proposition \ref{decexp} nous avons qu'il existe $\tilde{p}$ indépendant de $\gamma$ et $c(p)$ telles que 
$$P(e\in \mathcal{P}_{t+s}|e\notin \mathcal{P}_t\cup \mathcal{I}_t) \leqslant \sum_{n,m,\gamma} \exp(-c(p)(n-m)).
$$
Il nous faut maintenant compter le nombre de chemin $\gamma$ de longueur $n$ avec $m$ changements de temps entre $t$ et $t+s$. Or le chemin est monotone, il y a $\displaystyle s \choose m$ choix pour les instants du chemin. Or $\gamma$ est à l'intérieur de $B(e,l/2)$, il y a au plus $l^{md}$ choix pour les arêtes de changement de temps. Le nombre de $\gamma$ est donc majoré par
$$ {s\choose m} l^{md} (2d-1)^{n-m}\leqslant (sl^d)^m(2d-1)^{n-m}.
$$
Or $m\leqslant s$ et $n\geqslant l/2$, nous obtenons donc une majoration de la probabilité suivante
\begin{multline*}P(e\in \mathcal{P}_{t+s}|e\notin \mathcal{P}_t\cup \mathcal{I}_t) \leqslant \sum_{n\geqslant l/2}s(sl^d)^s\exp(-c(p)(n-s))\\\leqslant \sum_{n\geqslant l/2} \exp(\ln s+s\ln s+sd\ln l +sc(p)-c(p)n)
\\ \leqslant \exp(\ln s+s\ln s+sd\ln l +sc(p) -c(p)\frac{l}{3}).
\end{multline*}
Or il existe une constante $K$ telle que $\displaystyle s \leqslant K(\frac{l}{\ln l})$, nous pouvons majorer $\ln s+s\ln s+sd\ln l +sc(p)$ par $\displaystyle 3Kl+Kc(p)\frac{l}{\ln l}$. Or $c(p)$ tend vers $+\infty$ quand $p$ tend vers $1$, donc pour $l$ assez grand et $p>\tilde{p}$, nous avons 
$$P(e\in \mathcal{P}_{t+s}|e\notin \mathcal{P}_t\cup \mathcal{I}_t) \leqslant \exp(-c(p)l).
$$

Le cas où $e\in \mathcal{I}_{t+s}$ se déduit facilement du cas $e\in \mathcal{P}_{t+s}$. En effet, pour avoir une interface contenant l'arête $e$ à l'instant $t+s$, il faut que l'arête $e$ devienne pivot avant cette instant. Sinon, elle se comporte identiquement dans les deux processus $X$ et $Y$. Donc nous avons l'inclusion des événements 
$$ \big\{ e\in \mathcal{I}_{t+s}, d(e,\mathcal{I}_t \cup \mathcal{P}_{t}\big) \geqslant l\} \subset \bigcup_{0<r<s}\big\{ e\in \mathcal{P}_{t+r}, d(e,\mathcal{I}_t \cup \mathcal{P}_{t}\big) \geqslant l\}.
$$
Or $s$ est majoré par $\displaystyle K\frac{l}{\ln l}$, nous avons la majoration de la probabilité
\begin{multline*} P(e\in \mathcal{I}_{t+s}, d(e,\mathcal{I}_t \cup \mathcal{P}_{t}\big) \geqslant l) \leqslant \sum_{0<r<s} P(e\in \mathcal{P}_{t+r}, d(e,\mathcal{I}_t \cup \mathcal{P}_{t})\geqslant l) \\ \leqslant K\frac{l}{\ln l}\exp(-c(p)l)\leqslant \exp(-c'(p)l)
\end{multline*}
pour une autre constante $c'(p)$. 
\end{proof}

\end{document}