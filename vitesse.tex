\documentclass[titlepage,a4paper,12pt]{article}

\usepackage{amsmath,amsfonts,amssymb,amsthm,mathrsfs,stmaryrd,enumitem}
\usepackage[french]{babel}
\usepackage{tikz}
\usepackage[utf8]{inputenc}
\usepackage{centernot}
\setlength{\parindent}{0cm}
\newcounter{def}
\newcounter{th}
\newcounter{propo}
\newcounter{coro}
\frenchbsetup{StandardLists = true}


\newtheorem{defi}[def]{Définition}
\newtheorem{thm}[th]{Théorème}
\newtheorem{prop}[propo]{Proposition}
\newtheorem{cor}[coro]{Corollaire}


\setlength{\parskip}{0pt}
\begin{document}
\section{Introduction}
\subsection{Les notations générales}
\paragraph{Le réseau $(\mathbb{Z}^d,\mathbb{E}^d)$.} Soit $x,y$ deux points de $\mathbb{Z}^d$, nous disons que $x,y$ sont voisins s'ils sont à distance 1 en norme euclidienne. L'ensemble $\mathbb{E}^d$ est l'ensemble des paires $\{x,y\}$, où $x,y$ sont deux points voisins de $\mathbb{Z}^d$. Le réseau $(\mathbb{Z}^d, \mathbb{E}^d)$ est le graphe dont $\mathbb{Z}^d$ est l'ensemble des sommets et $\mathbb{E}^d$ est l'ensemble des arêtes. Soit $A$ un sous-ensemble de $\mathbb{R}^d$, nous disons que $e = \langle x,y\rangle$ est incluse dans $A$ si le segment ouvert $]x,y[$ est inclus dans $A$.

\paragraph{Les boîtes $\Lambda(\ell,h)$.} Nous notons $\Lambda(\ell,h)$ le sous graphe de $(\mathbb{Z}^d,\mathbb{E}^d)$ à l'inté\-rieur du rectangle $[-\ell,\ell]\times[-h,h]$. Nous notons aussi $T(\ell,h)$ le bord supérieur de $\Lambda(\ell,h)$, i.e. le segment $[-\ell,\ell]\times(0,h)$. Nous notons $B(\ell,h)$ le bord inférieur de $\Lambda(\ell,h)$, i.e. $[-\ell,\ell]\times(0,-h)$. Les deux bords verticaux de la boîte $\Lambda(\ell,h)$ sont notés $V(\ell,h)$.
\begin{figure}[h]
\center
\begin{tikzpicture}
\draw (0,0) rectangle (8,6);
\node at (4,3) {$\Lambda(\ell,h)$};
\node[above] at (4,6) {$T(\ell,h)$};
\node[below] at (4,0) {$B(\ell,h)$};
\node[left] at (0,3) {$V(\ell,h)$};
\node[right] at (8,3) {$V(\ell,h)$};
\end{tikzpicture}
\caption{La boîte $\Lambda(\ell,h)$}
\end{figure}

\paragraph{Les ensembles séparants.} Soient $A,B$ deux sous-ensembles de $\mathbb{Z}^d$. Nous disons qu'un ensemble d'arêtes $S\subset \mathbb{E}^d$ sépare $A$ et $B$ si aucune partie connexe du graphe $(\mathbb{Z}^d,\mathbb{E}^d \setminus S)$ n'intersecte simultanément $A$ et $B$. Un tel ensemble est appelé un ensemble séparant pour $A$ et $B$. Nous disons que $S$ est un ensemble séparant minimal pour $A,B$ si aucun sous-ensemble strict de $S$ ne sépare $A$ et $B$.

\paragraph{Les configurations.} L'espace des configurations est $\Omega = \{0,1\}^{\mathbb{E}^d}$. Une configuration générique est notée $\omega = \{\omega(e), e\in \mathbb{E}^d\} \in \Omega$ . L'arête $e$ est ouverte dans la configuration $\omega$ si $\omega(e) = 1$ et fermée si $\omega(e)= 0$. 
Soient $A$ un sous-ensemble de $(\mathbb{Z}^d,\mathbb{E}^d)$ et $\omega$ une configuration, la configuration $\omega$ restreinte à $A$, notée $\omega\mid_{A}$, est la restriction de $\omega$ aux arêtes dont les 2 extrémités sont incluses dans $A$.
Soient $e\in \mathbb{E}^d$ une arête et $\omega\in \Omega$ une configuration, nous définissons les configurations $\omega^e,\omega_e$ par:
$$
 \forall f\in \mathbb{E}^d \qquad\omega^e(f) = \left\lbrace \begin{array}{cc}
	\omega(f) & f\neq e \\
	1 & f=e 
 \end{array}\right., \quad
 \omega_e(f) = \left\lbrace \begin{array}{cc}
	\omega(f) & f\neq e \\
	0 & f=e
 \end{array}\right..
$$
Les configurations $\omega^e,\omega_e$ sont obtenues à partir de $\omega$ en ouvrant ou fermant l'arête $e$.
\paragraph{Les chemins.} Soient $x$ et $y$ deux sommets dans $\mathbb{Z}^d$, un chemin entre $x$ et $y$ est une suite $x_0,e_0, x_1,e_1,\dots, e_n,x_{n+1}$ de sommets $x_i$ et d'arêtes $e_i$ distincts où $x_0=x$ et $x_{n+1}=y$ et $e_i$ est l'arête joignant $x_i$ à $x_{i+1}$. 
Nous disons que $x$ et $y$ sont reliés par un chemin ouvert dans la configuration $\omega$, noté $\smash{x\overset{\omega}{\longleftrightarrow }y}$, s'il existe un chemin qui relie $x$ et $y$ tel que toute arête de ce chemin est ouverte dans la configuration $\omega$. Nous notons $\smash{x\overset{\omega}{\longarrownot\longleftrightarrow} y}$ l'événement complémentaire. Soit $A$ un sous-ensemble de $\Lambda(l,h)$, nous notons $\smash{x\overset{A}{\longleftrightarrow} y}$ s'il existe un chemin ouvert entre $x$ et $y$ dont les arêtes sont incluses dans $A$.
Soient $A,B$ deux sous-graphes de $(\mathbb{Z}^d,\mathbb{E}^d)$, nous disons que $A$ est connecté à $B$ dans la configuration $\omega$, ce que nous notons $\smash{A\overset{\omega}{\longleftrightarrow} B}$, s'il existe un chemin ouvert entre un sommet de $A$ et un sommet de $B$.
\paragraph{La probabilité de percolation.} Soit un réel $p\in [0,1]$. Sur l'espace $\Omega$, nous considérons la tribu cylindrique $\mathcal{F}$. Nous considérons la probabilité produit $$P_p = (p\delta_1 +(1-p)\delta_0)^{\otimes\mathbb{E}^d}.$$ Intuitivement, une configuration s'obtient en fermant indépendamment cha\-que arête du réseau $\mathbb{Z}^d$ avec une probabilité $1-p$. Pour $A$ un sous-ensemble de $(\mathbb{Z}^d,\mathbb{E}^d)$, $\Omega_{|A}$ est l'ensemble des configurations restreintes dans $A$.
\subsection{Le processus de percolation dynamique}
Nous considérons le processus de percolation dynamique dans une boîte $\Lambda$. Il s'agit d'une suite de triplets $(X_t,E_t,B_t)_{t\in \mathbb{N}}$, où $(X_t)_{t\in\mathbb{N}}$ est un processus à valeurs dans $\Omega_{|\Lambda}$, $(E_t)_{t\in\mathbb{N}}$ est une suite d'arêtes dans la boîte $\Lambda$ et $(B_t)_{t\in \mathbb{N}^*}$ est une suite de variables aléatoires à valeurs dans $\{0,1\}$. Décrivons maintenant la dynamique du processus. La suite $(E_t)_{t\in\mathbb{N}}$ est une suite indépendante d'arêtes aléatoires et chaque $E_t$ suit la loi uniforme sur l'ensemble des arêtes à l'intérieur de $\Lambda$. La suite $(B_t)_{t\in\mathbb{N}}$ est une suite indépendante de variables de Bernoulli de paramètre $p$. Le processus $(X_t)_{t\in\mathbb{N}}$ se définit par récurrence sur $t$, comme suit. Nous partons d'une configuration initiale $X_0$. A l'instant $t\geqslant 1$, nous changeons l'état de l'arête $E_t$ en $B_t$, i.e, nous définissons
$$ X_t(e) = \left\lbrace \begin{array}{cc}
X_{t-1}(e) & \text{si }e\neq E_t \\
B_{t} & \text{si } e = E_t
\end{array} \right..
$$ 

\paragraph{L'espace de probabilité de la percolation dynamique.}
Nous notons $$\overline{\Omega} = \Omega_{|\Lambda}^\mathbb{N}\times \Lambda^\mathbb{N}\times\{0,1\}^\mathbb{N}$$ l'espace de configuration de la percolation dynamique et $\overline{\mathcal{F}}$ la filtration naturelle associée. Nous notons $P_X$ la loi marginale du processus $(X_t)_{t\in\mathbb{N}}$ dans le triplet $(X_t,E_t,B_t)_{t\in\mathbb{N}}$.
\subsection{Les chemins espace-temps}
Nous introduisons d'abord les chemins espace-temps qui généralise les chemins. Un chemin espace temps est une suite de couples distincts $(e_i,t_i)_{i\in\mathbb{N}}$ qui vérifie la condition suivante, pour tout $i$ entier naturel nous avons soit $e_i = e_{i+1}$, soit $t_i = t_{i+1}\text{ et que }e_i \text{ et } e_{i+1} \text{ admet une extrémité en commun}$.
Nous introduisons maintenant les chemins espace-temps ouverts définis associés à un processus de percolation dynamique $X$. Nous définissons une relation $\smash{\overset{X}{\longleftrightarrow}}$ sur les couples $(e,t) \in (\mathbb{E}^d,\mathbb{N})$ de la manière suivante: soient $e = \langle a,b\rangle,f = \langle c,d\rangle \in \mathbb{E}^d$ deux arêtes, et $s,t\geqslant 0$ deux instants, nous posons
\begin{align*}
(e,t) \overset{X}{\longleftrightarrow} (f,t) &\quad\text{ si }
 a\overset{X_t}{\longleftrightarrow} c, X_t(e) = 1, X_t(f)=1\\
(e,s) \overset{X}{\longleftrightarrow} (e,t) &\quad\text{ si } \forall r \in [s\wedge t,s\vee t]\quad X_r(r)  = 1.
\end{align*} 
Nous considérons ensuite la relation d'équivalence sur les sommets de $\mathbb{Z}^d$ définie de la manière suivante: soient $x,y\in \mathbb{Z}^d, s,t\in \mathbb{N}$, nous posons
\begin{multline*}(x,s) \overset{X}{\longleftrightarrow} (y,t) \quad\text{ si}\quad\text{il existe une arête }
 e\text{ ayant }x\text{ pour extrémité},\\ \text{et une arête }f\text{ ayant }y\text{ pour extrémité }y \text{ telle que }
(e,s)\overset{X}{\longleftrightarrow} (f,t).
\end{multline*}
Finalement, nous disons qu'il existe un chemin espace-temps ouvert entre $(x,s)$ et $(y,t)$ si $\smash{(x,s) \overset{X}{\longleftrightarrow} (y,t)}$.

\subsection{Les interfaces via un couplage}
Nous introduisons un couplage qui nous permet de définir les interfaces entre le haut et le bas. 
Nous définissons le processus $(Y_t)_{t\in\mathbb{N}^*}$ à valeurs dans $\{0,1\}^{\mathbb{E}^d}$ par la formule suivante:
$$ Y_t(e) = \left\lbrace \begin{array}{cl}
X_t(e) & \text{si }e \neq E_t\\
0 &\text{si } e = E_t\text{ et } B_t = 0\phantom{T\overset{Y_{t-1}^{E_t}}{\longarrownot\longleftrightarrow} B}\\
1 & \text{si } e=E_t, B_t = 1 \text{ et } T\overset{Y_{t-1}^{E_t}}{\longarrownot\longleftrightarrow} B\\
0 & \text{si } e=E_t, B_t = 1 \text{ et } T\overset{Y_{t-1}^{E_t}}{\longleftrightarrow} B
\end{array} \right.
$$
Le processus $Y$ est un processus de percolation dynamique conditionné par l'événement $\{T\longarrownot\longleftrightarrow B\}$. Nous notons $P_Y$ la loi de ce processus et $P$ la loi du couple $(X,Y)$. Lorsque nous essayons d'ouvrir une arête $e$ fermée de la boîte $\Lambda$ à un instant, nous vérifions d'abord si cette modification induit une connexion par un chemin ouvert entre le haut et le bas de la boîte dans le processus $Y$. Si c'est le cas, l'arête $e$ reste fermée dans le processus $Y$ mais s'ouvre dans le processus $X$. Par contre, la fermeture d'une arête ne peut pas créer une connexion entre le haut et le bas dans le processus $Y$, et les deux processus $X$ et $Y$ se comportent identiquement lorsque nous essayons de fermer une arête.
Nous proposons maintenant une définition de l'interface à partir du couplage précédent.
\begin{defi} Soit $X_t$ le processus de percolation dynamique et $Y_t$ le processus de percolation dynamique conditionné par $\{T\longarrownot\longleftrightarrow B\}$. L'interface à l'instant $t$ entre $T$ et $B$, que nous notons $\mathcal{I}_t$, est l'ensemble des arêtes de $\Lambda$ qui sont différentes dans les configurations $X_t$ et $Y_t$,i.e,
$$ \mathcal{I}_t = \big\{\,e\in\Lambda : X_t(e) \neq Y_t(e)\,\big\}
$$
\end{defi}
Les arêtes de $\mathcal{I}_t$ sont fermées dans la configuration $Y_t$ mais ouvertes dans $X_t$, car la configuration $X_t$ est au-dessus de la configuration $Y_t$.
Nous définissons aussi l'ensemble $\mathcal{P}_t$ des arêtes pivot dans $\Lambda$ pour l'événement $\{T\longleftrightarrow B\}$ dans la configuration $Y_t$. 
\begin{defi} L'ensemble $\mathcal{P}_t$ est l'ensemble des arêtes dans $\Lambda$ dont l'ouverture induit l'événement $\{T\longleftrightarrow B\}$ dans la configuration $Y_t$, i.e,
$$\mathcal{P}_t = \big\{ \,e\in\Lambda : T\overset{Y_t^e}{\longleftrightarrow} B\,\big\}
$$
\end{defi}
\section{La construction d'un chemin espace-temps}
Nous allons énoncer une première proposition qui permet de construire un chemin espace -temps dont nous avons besoin pour la suite de notre étude.

\begin{prop} \label{stc}Soient $0<s<t$ deux instants et $e$ une arête de $ \mathcal{P}_t$ et $f$ une arête de $\mathcal{P}_s$. Nous supposons que $\mathcal{P}_r$ est non vide à tout instant $r$ entre $s$ et $t$. Il existe un chemin espace-temps monotone en temps et fermé dans le processus $Y$ qui relie $(e,t)$ à $(f,s)$. De plus, si une arête $e$ est visitée à deux instants différents $u$ et $v$, alors $e$ est ouverte à un instant $w$ dans l'intervalle $]u,v[$.
\end{prop}

\begin{proof}
Nous remarquons d'abord qu'à l'instant $t$, il existe un chemin fermé qui relie toutes les arêtes pivot. Nous considérons $\theta(t)$ le plus ancien instant où il existe une arête de $\mathcal{P}_t$ qui devient pivot, i.e.,
\begin{multline*} \theta(t) = \inf\big\{ \,r\leqslant t, \exists e_1 \in \mathcal{P}_t, e_1 \text{ devient pivot à l'instat } r \\\text{ et }e_1\text{ reste pivot entre }s\text{ et }t \big\}.
\end{multline*}
Nous montrons que $\theta(t)< t$. En effet, si à l'instant $t$, il y a une arête qui se ferme, alors il n'y a pas d'arête pivot créée à l'instant $t$, donc les arêtes de $\mathcal{P}_t$ sont aussi pivot à l'instant $t-1$. Dans le cas contraire, supposons qu'une arête s'ouvre à l'instant $t$. Nous considérons une arête $\varepsilon$ de $\mathcal{P}_{t-1}$, il existe un chemin passant par $\varepsilon$ relie le haut et le bas de $\Lambda$ à l'instant $t-1$, dont toutes les arêtes, hormis $\varepsilon$ sont ouvertes. Or l'ouverture d'une arête autre que $\varepsilon$ ne modifie pas ce chemin ouvert, l'arête $\varepsilon$ est toujours pivot à l'instant $t$. Nous avons donc trouvé une arête qui est devenue pivot avant l'instant $t$. D'où $\theta(t)< t$.

Si $\theta(t)\leqslant s$, le chemin $(e,t),(e_1,t),(e_1,s),(f,s)$ est un chemin espace-temps qui répond à la question. Supposons, $\theta(t)> s$. Nous considérons l'arête-temps $(e_1,\theta(t))\in \mathcal{P}_{\theta(t)}$ qui est reliée à $(e,t)$ par le chemin $(e,t),(e_1,t),(e_1,\theta(t))$. Nous devons maintenant montrer le résultat voulu avec $(e_1,\theta(t))$ à la place de $(e,t)$. Nous répétons la procédure précédente et nous construisons une suite décroissante d'instants $(t,\theta(t),\theta(\theta(t)),\dots)$ et une suite d'arêtes $(e,e_1,e_2,\dots)$. Or tout instant de la suite est supérieur à $s$, la suite est forcément finie. Donc il existe un $k\in\mathbb{N}$ tel que $\theta^k(t)\leqslant s$. Or pour tout $0\leqslant i < k$, l'arête $(e_i,\theta^i(t))$ est reliée à $(e_{i+1},\theta^{i+1}(t))$ par un chemin espace-temps fermé décroissant en temps. Nous obtenons un chemin espace-temps fermé qui relie $(e,t)$ à $(e_k,\theta^k(t))$ en concaténant les bouts obtenus à chaque étape. Par construction, ce chemin est monotone en temps. 

Pour prouver le deuxième point de la proposition, nous faisons un procédure de modification par récurrence. Nous notons le chemin obtenu à la fin de la procédure précédente par $(e_i,\theta^i(t))_{0\leqslant i \leqslant k}$ avec $e_0 = e,t_0 = t$ et $e_k=f, t_k = s$. Nous commençons avec l'arête $e_0$. Lorsque nous arrivons à l'arête $e_i$, nous distinguons différents cas: 
\begin{itemize}[leftmargin = 0.4cm]
\item Si pour tout indice $j$ tel que $i<j\leqslant k$ nous avons $e_j\neq e_i$, nous passons $e_i$ et nous recommençons la modification avec $e_{i+1}$.
\item S'il existe $i<j\leqslant k$ un indice tel que $e_j=e_i$ et que $e_i$ s'ouvre entre $\theta^j(t)$ et $\theta^i(t)$, nous passons $e_i$ et nous recommençons la modification avec $e_{i+1}$.
\item S'il existe $i<j\leqslant k$ un indice tel que $e_j=e_i$ mais que $e_i$ reste fermée entre $\theta^j(t)$ et $\theta^i(t)$, nous enlevons les arêtes entre $i$ et $j$, nous relions directement $e_i$ et $e_j$ par un changement de temps et nous recommençons avec $e_j$. 
\end{itemize}
Or la longueur du chemin diminue à chaque modification et donc procédure se termine. Comme cette procédure ne modifie pas l'ordre des $\theta^i$, la monotonie est conservée.
\end{proof}

\section{La comparaison entre deux processus}
Nous étudions maintenant le chemin espace-temps dans les deux processus. En particulier, nous comparons les états des arêtes fermées du chemin dans chaque processus.
Soit $(e,t)$ une arête-temps fermée dans le processus $Y$. Dans le processus $X$, vu qu'il n'y a pas de conditionnement, elle peut être ouverte ou fermée à l'instant $t$. Si elle est ouverte, par la définition de l'interface, elle est dans $ \mathcal{I}_t$. Dans le cas contraire, elle est aussi fermée dans la configuration $X_t$. Si $e$ se ferme dans $Y$ à un instant, elle se ferme aussi dans $X$ au même instant. Par contre, si $e$ est pivot dans $Y$, elle peut s'ouvrir dans $X$ mais rester fermée dans $Y$. Maintenant, regardons le chemin espace-temps obtenu dans la proposition \ref{stc}. Comme il existe des arêtes pivot dans ce chemin, celles-ci ne sont pas forcément fermé dans le processus $X$. Par contre, nous pouvons considérer le dernier instant de fermeture avant le passage sur une arête. Plus précisément, notons $(e_i,t_i)_{1\leqslant i \leqslant n}$ le chemin espace-temps. Nous considérons $\tilde{t_i}$ le dernier instant d'arrivée du processus de fermeture de l'arête $e_i$ avant $t_i$, i.e,
$$ \widetilde{t}_i = \sup \big\{\, t\leqslant t_i: E_{t} = e_i, B_t = 0\,\big\}
$$
Nous allons distinguer trois cas. Dans le premier cas $\widetilde{t_i}\geqslant s$, l'arête $e_i$ est fermée à $t_i$ car il y a une fermeture après l'instant $s$ et pas d'ouverture après cette fermeture. Dans le deuxième cas $\widetilde{t_i}< s$ et $e_i$ est fermé à l'instant $s$ dans les deux processus. Dans le dernier cas, $\widetilde{t_i}< s$ et à l'instant $s$, $e_i$ est fermée dans $Y$ mais ouverte dans $X$, nous remarquons qu'alors $e_i \in \mathcal{I}_s$. Nous en déduisons la proposition suivante:

\begin{prop}\label{couple} Soient $s<t$ deux instants et $\Gamma$ un chemin aléatoire espace-temps fermé entre $s$ et $t$ dans le processus conditionné $Y$. Toute arête $e$ de $\Gamma$ vérifie une des trois propriétés suivantes:
\begin{itemize}[leftmargin = 1cm]
\item[(i)] Il existe un instant d'arrivée $r\in ]s,t]$ du processus de fermeture en $e$, i.e., il existe un instant $r\in]s,t]$ tel que $E_r = e$ et $B_r = 0$;
\item[(ii)] $e$ est fermée à l'instant $s$ dans le processus $X$ à l'instant $s$;
\item[(iii)] $e$ est dans l'interface à l'instant $s$.

\end{itemize}
\end{prop}


\section{Une inégalité de type BK}
Nous allons montrer une inégalité qui nous permettra de comparer un chemin espace-temps dans les deux processus $X$ et $Y$. Nous reprenons les notations de la proposition \ref{couple} et nous définissons l'événement $$ \rm{Fermeture}(e,]s,t]) = \big\{\, \exists r \in ]s,t] \quad E_r = e,B_r = 0\,\big\}.
$$
Nous définissons la projection en espace d'un chemin espace-temps comme suit. Soit $\gamma = (e_i,t_i)_{1\leqslant i \leqslant n}$ un chemin espace temps, la projection en espace de $\gamma$, que nous notons $\rm{Espace}(\gamma)$, est l'ensemble d'arêtes obtenu en supprimant les deuxièmes arêtes des changements de temps dans la suite $(e_i)_{1\leqslant i \leqslant n}$,i.e.
$$\mathrm{Espace}(\gamma) = \{\,e\in \gamma\,\} \setminus \{\,e_i\in \gamma, e_{i-1} = e_i\,\}
$$
Soit $x,y$ deux sommets de $\mathbb{Z}^d$, notons $\mathcal{E}_Z(x,y),\widetilde{\mathcal{E}}(x,y)$ et $\mathcal{D}$ les ensembles de configurations dans $\Lambda$ entre $s$ et $t$ suivants:
\begin{align*}
\mathcal{E}_X(x,y,s,t)&=\left\lbrace\begin{array}{c}\exists \gamma
\text{ un chemin espace-temps fermé dans }X\\\text{ qui relie }(x,s),(y,t)\text{ sur l'intervalle de temps } [s ,t]\\\text{ et disjoint de }\mathcal{P}_s\cup \mathcal{I}_s \text{ à l'instant }s
\end{array} \right\rbrace\\
\widetilde{\mathcal{E}}(x,y,s,t)&=\left\lbrace \begin{array}{c}
 \exists\gamma\text{ un chemin espace-temps qui relie } (x,s),(y,t)\\ \text{ tel que }\forall e \in \mathrm{Espace}(\gamma), X_s(e) = 0 \text{ ou } \rm{Fermeture}(e,]s,t]) 
\end{array}\right\rbrace\\
\mathcal{D}(s,t)& = \{\,T\longarrownot\longleftrightarrow B \text{ entre }s \text{ et }t\text{ dans le processus }X\,\}
\end{align*}
De la même manière que $\mathcal{E}_X(x,y,s,t)$ nous définissons l'ensemble $\mathcal{E}_Y(x,y,s,t)$.
Nous montrons l'inégalité suivante:
\begin{prop} \label{BK'} Pour tout sommets $x,y$ et instants $s<t$, nous avons
$$P\left(\mathcal{E}_X(x,y,s,t)\cap\mathcal{D}(s,t)\right)\leqslant P\left(\widetilde{\mathcal{E}}(x,y,s,t)\right)P\left(\mathcal{D}(s,t)\right).$$
\end{prop}
\begin{proof}
Nous utilisons le couplage et nous considérons l'événement $\mathcal{E}$. Comme le processus $Y$ est le processus $X$ conditionné par $\mathcal{D}$, nous obtenons:
$$P(\mathcal{E}_Y(x,y,s,t)) = P\left(\mathcal{E}_X(x,y,s,t)\,\big|\,\mathcal{D}(s,t)\right) =  \frac{P\left(\mathcal{E}_X(x,y,s,t)\cap\mathcal{D}(s,t)\right)}{P(\mathcal{D}(s,t))}.
$$
Sur l'événement $\mathcal{E}_Y(x,y,s,t)$ les arêtes du chemin espace temps $\gamma$ sont disjointes de $\mathcal{P}_s$, par la proposition \ref{couple}, elles vérifient donc $(i)$ ou $(ii)$ et l'événement $\widetilde{\mathcal{E}}(x,y,s,t)$ est réalisé dans le processus $X$. Nous obtenons
$$ P(\mathcal{E}_Y(x,y,s,t))\leqslant P\left(\widetilde{\mathcal{E}}(\gamma)\right).
$$
Enfin, nous obtenons l'inégalité voulue en combinant les deux inégalités précédentes.
\end{proof}
Nous remarquons que cette inégalité permet de majorer la probabilité de l'intersection de deux événement par un produit de deux probabilités, comme l'inégalité de BK. Mais la différence avec l'inégalité BK est que l'un des deux événements à droite de l'inégalité est modifié selon le couplage. 
\section{La vitesse du déplacement}
Nous présentons le résultat central dans cette partie. Nous notons $d$ la distance Euclidienne dans $\mathbb{R}^d$. Nous identifions une arête avec le segment fermé qui relie ses extrémités.
\begin{thm}\label{vitesse}
Il existe deux constantes $c,K$ strictement positives telles que, pour toute arête $e$, conditionnellement à l'événement $d(e,\mathcal{P}_t\cup\mathcal{I}_t)\geqslant \ell$, la probabilité pour que $e$ appartienne à $\mathcal{I}_{t+s}\cup \mathcal{P}_{t+s}$ décroît exponentiellement vite avec la distance $\ell$ pour les $s$ entre $0$ et $K\ell$, i.e,
\begin{align*}
&\exists c > 0,\,\exists K > 0,\, \forall t>0,\, \forall \ell> 0,\,\forall s> 0,\, 0< s< K\ell\\ &P\big(e\in \mathcal{P}_{t+s}\cup \mathcal{I}_{t+s}\,\big|\,d(e,\mathcal{P}_t\cup \mathcal{I}_t\big) \geqslant \ell)\leqslant \exp(-c\ell)
\end{align*}
\end{thm}
Pour montrer ce résultat, nous allons d'abord choisir deux sommets $x,y$ et construire un chemin espace-temps à l'aide de la proposition \ref{stc} qui est induit par le déplacement de l'interface. Ensuite, pour les deux extrémités fixées, nous appliquons la proposition \ref{BK'}. Enfin, nous estimons la probabilité de $\widetilde{\mathcal{E}}(x,y,s,t)$ et nous sommons sur les $x,y$ possibles. Nous commençons par calculer $P(\widetilde{\mathcal{E}}(x,y,s,t))$. 
\begin{prop} \label{decexp}
Soit $S$ la projection en espace d'un chemin espace-temps obtenu par la proposition \ref{stc}. Nous supposons qu'il existe une constante $\kappa$ telle que $t-s \leqslant \kappa |\Lambda|$, alors il existe une constante $0\leqslant\widehat{p}< 1$ et une constante $c(p)>0$ qui tend vers $+\infty$ quand $p$ tend vers $1$ telles que $$\forall p \geqslant \widehat{p}, \quad P\left(\forall e\in S, \mathrm{Fermeture}(e,]s,t]) \text{ ou }X_s(e) = 0\right)\leqslant \exp(-c(p)|S|).$$
\end{prop}
\begin{proof}
Nous fixons d'abord une arête $e\in \Lambda$. Nous calculons la probabilité que $e$ vérifie $(B)$ ou $(C)$.
\begin{multline*}P\left( \mathrm{Fermeture}(e,]s,t])\text{ ou }X_s(e) = 0 \right)\leqslant P(X_s(e) = 0)\\+P(\exists r \in ]s,t], E_r = e, B_r = 0)
\end{multline*}
Or la loi de la configuration $X_s$ dans $\Lambda$ à l'instant $s$ est la loi de la percolation Bernoulli de paramètre $p$, nous obtenons $$P_X(e \text{ fermée à l'instant } s) = 1-p.$$
Comme à chaque instant, nous choisissons indépendamment une arête dans $\Lambda$ uniformément et nous tirons indépendamment l'état de cette arête selon la loi de Bernoulli de paramètre $p$, nous avons pour la deuxième probabilité
$$ P(\exists r \in ]s,t], E_r = e, B_r = 0) \leqslant \frac{t-s}{|\Lambda|}(1-p)\leqslant \kappa(1-p).
$$
Nous obtenons que pour une arête $e$ fixée, $$P\left( \mathrm{Fermeture}(e,]s,t])\text{ ou }X_s(e) = 0 \right)\leqslant (1+\kappa)(1-p).$$
Nous considérons maintenant toutes les arêtes de $S$. Pour toute arête $e$, l'événement $\mathrm{Fermeture}(e,]s,t])$ est entièrement déterminée par le processus $X$ où les arêtes distinctes sont indépendantes. Par la construction de $\gamma$, s'il existe deux arêtes $e,f$ de $S$ telles que $e = f$, alors il existe un instant entre les deux visites où l'arête $e$ est ouverte. Donc $e$ et $f$ sont fermées indépendamment. Nous pouvons donc factoriser $$P(\forall e\in S, \mathrm{Fermeture}(e,]s,t])\text{ ou } X_s(e) =0)$$ par le produit des $P(\mathrm{Fermeture}(e,]s,t])\text{ ou } X_s(e) =0)$ et nous obtenons 
\begin{multline*}P\left(\forall e\in S, \mathrm{Fermeture}(e,]s,t])\text{ ou }X_s(e) = 0\right)\leqslant (1+\kappa)^{|S|}(1-p)^{|S|}\\ = \exp\big(\ln[(1+\kappa)(1-p)](|S|)\big).
\end{multline*}
Nous posons $c(p) = -\ln[(1+\kappa)(1-p)]$. Or $\ln[(1+\kappa)(1-p)]$ tend vers $-\infty$ quand $p$ tend vers 1. Il existe une constante $\hat{p}<1$ telle que pour tout $p\geqslant \hat{p}$, nous avons $c(p)>0$. Nous obtenons aussi 
$ \lim_{p\rightarrow 1}c(p) = +\infty
$.
\end{proof}
Nous montrons maintenant le théorème \ref{vitesse}. 
\begin{proof}
Nous considérons d'abord le cas où l'arête $e$ est dans $\mathcal{P}_{t+s}$ et est à distance au moins $\ell$ de $\mathcal{P}_t\cup \mathcal{I}_t$.
D'après la proposition \ref{stc}, il existe un chemin espace-temps $\gamma$ qui relie $e$ à une arête de $\mathcal{P}_t$. Nous notons $\gamma= (e_i,t_i)_{1\leqslant i \leqslant N}$ avec $e_1 = e, t_1 = t+s$ et $e_N\in \mathcal{P}_t, t_N = t$. Ce chemin est de projection en espace au moins $l$ car $d(e,\mathcal{P}_t)\geqslant \ell$. Nous notons $(e_n,t_n)$ la première arête de $\gamma$ telle que la longueur de la projection en espace de $(e_1,t_1),\dots,(e_n,t_n)$ dépasse $\ell/2$, i.e. 
$$ n = \inf\,\left\lbrace k>1, |\mathrm{Espace}((e_i,t_i)_{1\leqslant i \leqslant k})| \geqslant \frac{\ell}{2} \right\rbrace.
$$
Nous notons $\rho = (e_1,t_1),\dots,(e_n,t_n)$. Nous notons aussi $x$ le sommet de $e_1$ qui n'est pas dans $e_2$ et $y$ le sommet de $e_n$ qui n'est pas dans $e_{n-1}$. Remarquons que nous avons nécessairement $e_{n-1} \neq e_n$ car $n$ est le plus petit indice qui vérifie $|\mathrm{Espace}((e_1,t_1),\dots(e_n,t_n))|\geqslant \ell/2$. 
Nous écrivons donc 
\begin{multline*} P(e\in \mathcal{P}_{t+s}|d(e,\mathcal{P}_t\cup \mathcal{I}_t)\geqslant \ell) \\\leqslant \sum_x\sum_{y\in B(x,\ell/2)}P( \exists \rho \text{ chemin espace-temps fermé dans }Y, x\overset{\rho}{\longleftrightarrow} y).
\end{multline*}
Nous fixons d'abord $S$ la projection en espace du chemin. La proposition \ref{couple} et \ref{BK'} nous permet de majorer chaque terme de la somme par $P(\widetilde{\mathcal{E}}(x,y,t,t+s))$ et d'après la proposition \ref{decexp} nous avons qu'il existe $\widetilde{p}$ indépendant de $\gamma$ et $c(p)$ telles que 
$$P(e\in \mathcal{P}_{t+s}|d(e,\mathcal{P}_t\cup \mathcal{I}_t)\geqslant \ell) \leqslant \sum_x\sum_{y\in B(x,\ell/2)}\sum_{S \subset B(x,\ell)} \exp(-c(p)|S|).
$$
Il nous faut maintenant compter le nombre de projections $S$ de longueur $\ell/2$ dans la boule de rayon $\ell$. En effet, nous pouvons le majorer par $(2d-1)^{|S|}$. Quitte à choisir $p$ assez proche de $1$, nous avons $c(p)> 2d-1$ et nous obtenons la majoration avec une nouvelle constante $c'(p) >0$,
$$P(e\in \mathcal{P}_{t+s}|d(e,\mathcal{P}_t\cup \mathcal{I}_t)\geqslant \ell) \leqslant \sum_x\sum_{y\in B(x,\ell/2)} \exp(-c'(p)|S|).
$$
Nous remarquons que $|S| = \ell/2$ et nous majorons enfin le nombre de sommet $y$ par $C(d)\ell^d$ avec $C(d)$ une constante dépend uniquement de la dimension et nous avons 
$$P(e\in \mathcal{P}_{t+s}|e\notin \mathcal{P}_t\cup \mathcal{I}_t) \leqslant 2C(d)\ell^d \exp(-c'(p)\frac{\ell}{2})\leqslant \exp(-c\ell)
$$
où $c$ est une constante strictement positive qui ne dépend que de $p$ et $d$ et qui tend vers $+\infty$ quand $p$ tend vers $1$.

Le cas où $e\in \mathcal{I}_{t+s}$ se déduit facilement du cas $e\in \mathcal{P}_{t+s}$. En effet, pour avoir une interface contenant l'arête $e$ à l'instant $t+s$, il faut que l'arête $e$ devienne pivot avant cette instant. Sinon, elle se comporte identiquement dans les deux processus $X$ et $Y$. Donc nous avons l'inclusion des événements 
$$ \big\{ e\in \mathcal{I}_{t+s}, d(e,\mathcal{I}_t \cup \mathcal{P}_{t}\big) \geqslant \ell\} \subset \bigcup_{0<r<s}\big\{ e\in \mathcal{P}_{t+r}, d(e,\mathcal{I}_t \cup \mathcal{P}_{t}\big) \geqslant \ell\}.
$$
Or $s$ est majoré par $K\ell$, nous avons la majoration de la probabilité
\begin{multline*} P(e\in \mathcal{I}_{t+s}, d(e,\mathcal{I}_t \cup \mathcal{P}_{t}\big) \geqslant \ell) \leqslant \sum_{0<r<s} P(e\in \mathcal{P}_{t+r}, d(e,\mathcal{I}_t \cup \mathcal{P}_{t})\geqslant \ell) \\ \leqslant K\ell\exp(-c\ell)\leqslant \exp(-c'\ell)
\end{multline*}
pour une autre constante $c'$ strictement positive et qui tend vers $+\infty$ quand $p$ tend vers $1$. 
\end{proof}

\end{document}