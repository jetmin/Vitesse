\documentclass[titlepage,a4paper,12pt]{article}

\usepackage{amsmath,amsfonts,amssymb,amsthm,mathrsfs,enumitem}
\usepackage[french]{babel}
\usepackage{comment}
\usepackage{tikz}
\usepackage[utf8]{inputenc}
\usepackage{centernot}
\usepackage[parfill]{parskip}
\newcounter{def}
\newcounter{thm}
\newcounter{prop}
\newcounter{cor}
\frenchbsetup{StandardLists = true}


\newtheorem{defi}[def]{Définition}
\newtheorem{thm}[thm]{Théorème}
\newtheorem{prop}[prop]{Proposition}
\newtheorem{cor}[cor]{Corollaire}

\newcommand{\nlongleftrightarrow}{\longleftrightarrow \kern-15pt\not \kern15pt}


\setlength{\parskip}{0pt}
\begin{document}
\section{Le modèle}
Nous notons $(X_t)_{t\geqslant 0}$ le processus de percolation dynamique sur le réseau $(\mathbb{Z}^d,\mathbb{E}^d)$ et nous considérons la restriction de $X$ dans une boîte finie $\Lambda$. Nous définissons la suite $(\tau_i)_{i\geqslant 0}$ des instants de changement dans une boîte finie $\Lambda(\ell,h)$, en posant $\tau_0 = 0$ et pour tout $i\geqslant 0$,
$$\tau_{i+1} \,=\, \inf \,\big\{ \, t>\tau_i: {X_t}\mid_{ \Lambda(\ell,h)} \neq {X_{\tau_i}}\mid_{ \Lambda(\ell,h)} \big \}.
$$
Pour tout $i\geqslant 1$, il existe une unique arête $e_i$ incluse dans $\Lambda(\ell,h)$ telle que $X_{\tau_i}(e_i) \neq X_{\tau_{i-1}}(e_i)$. Nous appelons la suite $(e_i)_{i\geqslant 1}$ la suite des arêtes modifiées.
Nous allons coupler $(Y_t)_{t\geqslant 0}$ à valeurs dans $\{0,1\}^{\mathbb{E}^d}$. D'abord nous posons $Y_0=X_0$ et $X_0$ une configuration qui vérifie $\{T\nlongleftrightarrow B\}$. Ensuite, soit $i\geqslant 0$, pour tout $s\in [\tau_i, \tau_{i+1}[$, nous posons $Y_s = Y_{\tau_i}$, et $Y_{\tau_{i+1}}(e) =Y_{\tau_i}(e)$ si $e\neq e_{i+1}$, et nous déterminons $Y_{\tau_{i+1}}(e_{i+1})$ en fonction de $X_{\tau_{i+1}}$ via la formule suivante:
$$Y_{\tau_{i+1}}(e_{i+1})=\left\lbrace \begin{array}{cl}
0 &\text{si }X_{\tau_{i+1}}(e_{i+1}) = 0\\
1 & \text{si }X_{\tau_{i+1}}(e_{i+1}) = 1, T \nlongleftrightarrow B \text{ dans }(Y_{\tau_i})^{e_{i+1}}\\
0 & \text{si }X_{\tau_{i+1}}(e_{i+1}) = 1, T \longleftrightarrow B \text{ dans }(Y_{\tau_i})^{e_{i+1}}
\end{array}\right..$$
Le processus $Y$ est le processus de percolation dynamique conditionné à rester dans l'ensemble $\{T\nlongleftrightarrow B\}$. Nous pouvons définir l'interface à l'aide de ce couplage.
\begin{defi}
Soit $(X_t,Y_t)_{t\geqslant 0}$ un couplage défini précédemment, nous définissons l'interface dans $\Lambda({\ell,h})$ au temps $t$, que nous notons $\mathcal{I}_t({\ell,h})$, comme l'ensemble aléatoire des arêtes qui sont ouvertes dans $X_t$ et fermées dans $Y_t$: $$ \mathcal{I}_t({\ell,h}) \,=\, \big\{ \,e\in \mathbb{E}^2: X_t(e) = 1, Y_t(e) = 0 \, \big\}.
$$
\end{defi}
Nous notons $\mathcal{P}_t$ l'ensemble des arêtes pivot pour l'événement $\{T\longleftrightarrow B\}$ dans $Y$.
\end{document}